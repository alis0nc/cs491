\documentclass[12pt,letterpaper]{article}
\usepackage{fullpage}
\usepackage{url}
\makeatletter
\newcommand*{\textoverline}[1]{$\overline{\hbox{#1}}\m@th$}

\newcommand*{\fitb}[2]
{\raisebox{-1.3ex}{\textoverline{{\makebox[#1]{{\rule{0pt}{5pt}\tiny #2}}}}}}
\makeatother
\begin{document}
\raggedright
\centerline{\textbf{High-Speed Networking and Distributed Applications}}
\centerline{\textbf{Lab Exercise 2: Introduction to GCF and Install Scripts}}
~\\
\centerline{\textbf{CS491 \hfill Summer 2013}}
\noindent\rule{\textwidth}{1pt}

\bigskip
In this lab exercise, you will set up the GENI Control Framework on your 
computer and work with resource specification (rspec) XML files. You will also 
load and run software automatically in a GENI node using install 
script functionality.


\subsection*{Procedure}

\begin{description}

\item[Step 0:]
Download \texttt{gcf} from  
\url{http://software.geni.net/local-sw/download&software=gcf-2.3.3.tar.gz} 
and follow the instructions at 
\url{http://trac.gpolab.bbn.com/gcf/wiki/QuickStart} to install it.\footnote
    {If you're using a syslab computer, you should copy the 
    \texttt{gcf-2.3.3} directory to your home folder because you don't have 
    root privileges on the box. If you're using your own laptop, you can copy 
    the directory anywhere you want.}
    
Stop right after step 3 ``Install software dependencies'', because 
configuration will be handled through the GENI Portal.

\item[Step 1:]
Go to your GENI Portal account and click the ``Profile'' link at the top 
navigation, then click ``Configure \texttt{omni}''. 

Download your customized configuration data, making sure you select 
\texttt{KU-CS491} as your default project.

Run \texttt{omni-configure.py}
in the terminal and \texttt{omni} will autoconfigure itself based on the 
configuration data you downloaded. If it complains that a key already exists, 
hit \texttt{n} to not replace it.

\item[Step 2:]
Create a slice through the GENI Portal and call it 
\texttt{\fitb{1in}{username}lab2}. Launch Flack.

\item[Step 3:]
Import the experiment. Select ``Import $\rightarrow$ Import from Web'' in
Flack, and supply this URL: 
\url{http://alis0nc.github.io/cs491/lab2/lab2.rspec}.
Specify that you want to use the Kettering aggregate manager, 
\texttt{geni.kettering.edu.cm}. Wait for your resources to become ready.

%%%%%%%%%%%%%%%%%%%%%%%%%%%%%%%%%%%%%%%%%%%%%%%%%%%%%%%%%%%%%%%%%%%%%%%%%%%%%%%
\newpage %%%%%%%%%%%%%%%%%%%%%%%%%%%%%%%%%%%%%%%%%%%%%%%%%%%%%%%%%%%%%%%%%%%%%%
%%%%%%%%%%%%%%%%%%%%%%%%%%%%%%%%%%%%%%%%%%%%%%%%%%%%%%%%%%%%%%%%%%%%%%%%%%%%%%%

\item[Step 4:]
Once they're ready, click on the info button for each node and add an install 
service. Supply the install url 
\url{http://alis0nc.github.io/cs491/lab2/lab2.tar.gz} and install location 
\texttt{/local}.

Add an execute service similarly with command 
\texttt{sudo /local/install-script.sh} to be executed using \texttt{sh}.

\item[Step 5:]
We are going to draw back the curtain and see what's going on behind the 
scenes. 
(\emph{Pay no attention to that man behind the curtain.}\footnote
    {\emph{The Wizard of Oz}, 1939.
     \url{www.youtube.com/watch?v=YWyCCJ6B2WE‎}})


In Flack, select ``View $\rightarrow$ Preview request document(s)''
and save the request rspec file. Examine it either in the Flack window 
or your favourite text editor. This xml file contains the specifications for 
the compute resources (virtual machine nodes) and network links you 
requested in Flack. This file gets sent to the aggregate manager so that it 
can give you what you requested.

\item[Step 6:]
We're switching from Flack to the command line \texttt{omni} tool now. 
In a terminal, run \texttt{omni.py createslice \fitb{1in}{username}lab2omni}.
This should succeed with a message to that effect in the terminal.

Now run
\texttt{omni.py createsliver -a ig-kettering \fitb{1in}{username}lab2omni
\fitb{2in}{path to rspec you saved in step 5}}. This takes a while, but after 
a few minutes you should see a ``Completed createsliver'' message and now your 
reserved nodes are ready to log into.

\item[Step 7:]
Follow the instructions at 
\url{http://groups.geni.net/geni/wiki/GEC17Agenda/GettingStartedWithGENI_II/ExecuteExperiment}
to log into your nodes and look at the experiment results. Use \texttt{ig-kettering}
for the \texttt{AM\_NICKNAME} and \texttt{\fitb{1in}{username}lab2omni} for the 
\texttt{SLICENAME}.

\item[Step 8:]
Clean up after yourself by deleting your resources. You already know how to do 
this in Flack. In omni, it's \texttt{omni.py -a ig-kettering deletesliver 
\fitb{1in}{username}lab2omni}. When that command succeeds, you'll see the 
output of ``Completed deletesliver''.

\end{description}

\subsection*{Acknowledgement}

Adapted from Wong, G. (2013-07-21).
``Getting started with GENI and the GENI Portal, part II''.
Presented at the \emph{17th GENI Engineering Conference}, Madison, Wisconsin.
\url{http://groups.geni.net/geni/wiki/GEC17Agenda/GettingStartedWithGENI_II}

\end{document}

