\documentclass[12pt,letterpaper]{article}
\usepackage{fullpage}
\usepackage{url}
\makeatletter
\newcommand*{\textoverline}[1]{$\overline{\hbox{#1}}\m@th$}

\newcommand*{\fitb}[2]
{\raisebox{-1.3ex}{\textoverline{{\makebox[#1]{{\rule{0pt}{5pt}\tiny #2}}}}}}
\makeatother
\begin{document}
\raggedright
\centerline{\textbf{High-Speed Networking and Distributed Applications}}
\centerline{\textbf{Lab Exercise 2: Introduction to the GENI APIs and} \texttt{omni}}
~\\
\centerline{\textbf{CS-491 \hfill Summer 2013}}
\noindent\rule{\textwidth}{1pt}

\bigskip
In this lab exercise, you will set up the GENI Control Framework on your 
computer and work with resource specification (rspec) XML files. You will also 
load and run software automatically in a GENI node using the rspec's install 
script functionality.


\subsection*{Procedure}

\begin{description}

\item[Step 0:]
Download \texttt{gcf} from  
\url{http://software.geni.net/local-sw/download&software=gcf-2.3.3.tar.gz} 
and follow the instructions at 
\url{http://trac.gpolab.bbn.com/gcf/wiki/QuickStart} to install it.\footnote
    {If you're using a syslab computer, you should copy the 
    \texttt{gcf-2.3.3} directory to your home folder because you don't have 
    root privileges on the box. If you're using your own laptop, you can copy 
    the directory anywhere you want.}
Stop right after step 3 ``Install software dependencies'', because 
configuration will be handled through the GENI Portal.

\item[Step 1:]
Go to your GENI Portal account and click the ``Profile'' link at the top 
navigation, then click ``Configure \texttt{omni}''. 

Download your customized configuration data, making sure you select 
\texttt{KU-CS491} as your default project.

Run 
\begin{verbatim}
omni-configure.py
\end{verbatim}
in the terminal and \texttt{omni} will autoconfigure itself based on the 
configuration data you downloaded. If it complains that a key already exists, 
hit \texttt{n} to not replace it.

\item[Step 2:]
Create a slice through the GENI Portal and call it 
\fitb{1in}{username}\texttt{lab2}. Launch Flack.

\item[Step 3:]
Import the experiment. Select ``Import $\rightarrow$ Import from Web'' in
Flack, and supply this URL: 
\url{http://alis0nc.github.io/cs491/lab2/lab2.rspec}.
Specify that you want to use the Kettering aggregate manager, 
\texttt{geni.kettering.edu.cm}.

\end{description}

\subsection*{Acknowledgement}

Adapted from Wong, G. (2013-07-21).
``Getting started with GENI and the GENI Portal, part II''.
Presented at the \emph{17th GENI Engineering Conference}, Madison, Wisconsin.
\url{http://groups.geni.net/geni/wiki/GEC17Agenda/GettingStartedWithGENI_II}

\end{document}

